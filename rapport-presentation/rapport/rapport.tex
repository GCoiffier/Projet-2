\documentclass{article}

\usepackage[T1]{fontenc}
\usepackage[utf8]{inputenc}
\usepackage[frenchb]{babel}
%\usepackage{fullpage}
\usepackage{graphicx}
%\usepackage{layout}
%\usepackage{geometry}
%\usepackage{setspace}
%\usepackage{soul}
%\usepackage{ulem}
%\usepackage{eurosym}
%\usepackage{bookman}
%\usepackage{charter}
%\usepackage{newcent}
%\usepackage{lmodern}
%\usepackage{mathpazo}
%\usepackage{mathptmx}
\usepackage{url}
%\usepackage{verbatim}
%\usepackage{moreverb}
%\usepackage{listings}
%\usepackage{fancyhdr}
%\usepackage{wrapfig}
\usepackage{color}
%\usepackage{colortbl}
\usepackage{amsmath}
\usepackage{amssymb}
\usepackage{amsfonts}
%\usepackage{mathrsfs}
%\usepackage{makeidx}
%\usepackage{parskip}
%\usepackage{titlesec}
\usepackage{hyperref}

% pour compiler: 

% faire    pdflatex ex-rapport
% (si les references aux numeros de parties apparaissent comme des
% "?", recompiler une fois)

% la compilation de la bibliographie est davantage une "incantation":
% faire     bibtex ex-biblio
% puis      pdflatex ex-rapport (un nombre premier de fois)


\title{ \Huge{Rapport Projet 2} \\ Où l'on parle de rongeurs}
\author{Guillaume Coiffier - Léo Valque}
\date{2017}

\newcommand{\non}[1]{\overline{#1}}


\begin{document}

\maketitle
\tableofcontents
\newpage

\section*{Avant Propos}

Nous avons programmé \textbf{en D}, en interfaçant notre programme
avec des morceaux d'assembleur que nous avons écrits à la main, entre
3 et 4 heures du matin uniquement sinon ça ne compte pas.

Nous exposons à la partie~\ref{s:orga} comment notre programme est structuré.

\section{Le projet}

\subsection{Remarques générales}

\subsection{}

\subsection{Organisation du code}
\label{s:orga}

Le code est structuré de la manière suivante~:
\begin{itemize}
\item bla
\end{itemize}

\section{Interpréteur fouine}

\subsection{Fouine}

\subsection{Le langage}
Coeur du langage + nos améliorations/extensions

\subsection{Structures de données}

\subsection{Les exceptions}

\subsection{Aspects impératifs et tableaux}

\section{La machine à pile SECD}

\subsection{Présentation}

\subsection{Implémentation}


\section{Interface et interprétation mixte}

\section*{Conclusion}

On constate que blibla.

\nocite{*}
\bibliographystyle{plain}
\bibliography{biblio}

\end{document}
