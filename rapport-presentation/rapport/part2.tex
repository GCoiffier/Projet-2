\part{Le projet}

\section{Organisation du code}

La plupart des fonctionnalités de notre projet sont encapsulées dans des modules. Ainsi, l'interpréteur fouine et la machine SECD constituent deux modules que l'on charge dans le fichier main. Nous avons privilégié cette approche pour deux raisons. D'une part par souci de propreté : on minimise le nombre de primitives que l'on peut appeler depuis l'extérieur du module à l'aide des mécanismes de signature (ce qui permet de plus d'inclure de la documentation dans les dites signatures), d'autre part pour rendre le code plus facilement maintenable, dans ce projet où certaines fonctions ont été réécrites plusieurs fois.

\subsection{Liste des fichiers}

\begin{description}
 \item[$\bullet$ main.ml :] Fichier principal. Lit les argument envoyés au programme et fait les différents appels aux différentes parties du code.
 \item[$\bullet$ fouine\_type.ml] Fichier contenant les définitions des types représentant un programme fouine dans Caml. Type unary\_op, binary\_op, variable, programme.
 \item[$\bullet$ interpreteur.ml :] Le fichier contenant le module au coeur de fouine. Contient ma grosse fonction d'interprétation d'un programme fouine, la fonction debug, qui permet d'afficher un programme fouine parsé, ainsi que de quoi réaliser l'exécution mixte.
 \item[$\bullet$ machine.ml :] Module implémentant la machine à pile.
 \item[$\bullet$ environnement.ml] La définition du module Environnement, construit à l'aide d'un dictionnaire.
 \item[$\bullet$ dictionnaire.ml] La classe de dictionnaire utilisée. Basée sur une table de hachage.
 \item[$\bullet$ lexParInterface.ml :] Fichier faisant l'interface entre les parser et le reste du code. Fournit simplement des primitives read\_prgm <fichier>.
 \item[$\bullet$ parser.mly , lexer.mll] Ces fichiers permettent de lire la formule donnée en entrée par le programme, et de construire un objet de type programme représentant le code à exécuter.
 \item[$\bullet$ parsMachine.mly , lexMachine.mll] Ces fichiers permettent de lire les instructions donnée en entrée par le programme, et de construire un objet de type instruction list représentant le code à exécuter sur la machine à pile.
\end{description}

\subsection{Liste des programmes fouine donnés en exemple}

\begin{itemize}
 \item exemples simples sur les différentes fonctionnalités de fouine (les noms des programmes sont a priori explicites)
 \item fibonacci récursif stupide
 \item fibonacci intelligent avec références
 \item factorielle
 \item tours de Hanoi
 \item crible d'erathosthène (illustration des tableaux)
\end{itemize}

\subsection{Bugs repérés mais non corrigés}

la traduction d'environnement fouine vers SECD suppose qu'une fonction est récursive. Du coup quelquechose défini par
\texttt{let f x = x in let f x = f x in f 2} ne s'exécutera pas correctement, il bouclera au lieu de renvoyer 2.

\section{L'interpréteur fouine} % section à remplir par Guillaume

\subsection{L'interprétation}

\paragraph{} L'interprétation est réalisée dans la fonction execute du fichier interpreteur.ml . Cette fonction prend en argument un programme fouine parsé (de type programme) et renvoie un entier. On utilise une fonction récursive auxiliaire qui associe une valeur de type 'ret' au programme. Ensuite, on appelle la petite fonction return qui renvoie un int à partir de ce ret.

\paragraph{} Ce type intermédiaire ret est nécessaire pour pouvoir utiliser des fonctions dans les programmes fouine. C'est le type des éléments qui sont associés à nos programmes dans l'environnement (Env.elt). Cela permet de renvoyer en interne autre chose que des entiers (ce qui est nécessaire pour les références et les fonctions). De plus, un filtrage par motif dans la fonction return permet de détecter les erreurs dynamiques (entier + fonction) ou (entier + ref) et d'obtenir des messages explicites.

\subsection{Structures de données}

\paragraph{} On utilisera des tables de hachage (Hastbl) qui associeront une valeur 'Env.elt' à une valeur 'programme'. Cette structure de donnée a le gros avantage de gérer parfaitement la portée des variables. En effet, d'après la documentation de Ocaml, lorsqu'une valeur y est assignée à la variable x dans une Hashtbl, l'ancienne valeur de x est remplacée par y. Lorsque l'on supprime l'association (x,y) dans la table, l'ancienne valeur associée à x est restaurée, ce qui est le comportement attendu.

\paragraph{} Cette structure de donnée supporte l'ajout d'un couple d'élément, la suppression d'un élement et de son association (avec éventuelle restauration de la précédente association), la recherche d'un élement et la copie (pour pouvoir faire des clotures).

\paragraph{} Elle contient la fonction transform\_env qui transforme l'environnement de l'interpréteur en code avec des Let et Letrec que peut comprendre la machine à pile et qui permet de transmettre de manière indirecte de l'environnement à la machine à pile.

\subsection{Fonctions et fonctions récursives}

\paragraph{} Pour l'implémentation des fonctions, on a ajouté un constructeur Cloture au type Env.elt.  Un cloture comprend l'expression de la fonction (le A de `let f x = A in`) et une copie de l'environnement au moment de la définition de la fonction. Cette copie est "brutale" : on copie l'intégralité de l'environnement sans chercher à savoir quelles valeurs sont inutiles.

\paragraph{} Dans le cas des fonctions récursives, on ajoute à la clôture crée... elle-même. Ainsi, on évite de faire autant de clôtures que d'appel récursif. Cela ne pose pas de problème tant qu'il existe un cas de sortie à la fonction récursive.

\paragraph{} Deux constructeurs sont associés aux fonctions dans le type programme :
    \begin{itemize}
     \item \texttt{Function\_def} : appellé lors de la définition de fonction, contient l'argument (unique) et l'expression de la fonction (qui peut elle même être une fonction).
     \item \texttt{Function\_call} : appellé lors de l'appel à une fonction, contient le nom de la fonction et la valeur de l'argument (qui est une expression non interprétée)
                        le nom de la fonction permet de retrouver la définition dans l'environnement. On interprète l'expression associée à la cloture en ajoutant à l'environnement de la cloture la valeur de l'argument.
    \end{itemize}


\subsection{Les exceptions}

\paragraph{} Les exceptions sont gérées grâce à l'ajout d'un booléen au type de retour de l'interpréteur. Ce booléen vaut vrai si une exception a été levée durant l'éxecution d'un bout de programme. Ainsi, seule l'instruction \texttt{raise} renvoie un booléen vrai. Le \texttt{try ... with} éxecute son premier argument normalement. Si une exception a été levée, alors on revient à l'ancien environnement et on exécute la partie \texttt{with ...} en prenant soin d'ajouter la valeur renvoyée par le raise à l'environnement.

\paragraph{} Tous les autres constructeurs ont été modifiés à cet effet : soit ils obtiennent récursivement un résultat sans exception, et tout se passe normalement, soit ce résultat renvoie une exception, et ils ne font alors que la propager. C'est ce qui permet à la valeur de l'exception de remonter jusqu'au dernier try.

\paragraph{} Si un raise est executé à l'extérieur d'un try, on vérifie que le booléen est vrai en sortie de la fonction d'évaluation, et donc on peut planter en annonçant une exception non rattrapée.

\paragraph{Historique de l'implémentation :} Les raises ont jadis été implémentés à l'aide du mécanisme de gestion d'exception de Ocaml (aux alentours du rendu 2). Comme cela ressemblait un peu à de la triche, nous avons, lors du rendu 3, implémenté une une gestion d'exceptions basé sur une pile d'environnement. Le soucis était que nous ne pouvions pas interrompre l'execution d'un code de cette façon. Par exemple, le code \texttt{try 3 + raise E 2 with E x -> 3;;} renvoyait 6, car le \texttt{raise} renvoyait le résultat du code situé après le \texttt{with} et ce qui se trouvait à l'intérieur du \texttt{try} continuait d'être exécuté. \\
Nous sommes donc passé à ce système de propagation de levée d'exception à l'aide de booléens qui allourdit certes la fonction d'interprétation, mais résout le problème.

\subsection{Aspects impératifs et tableaux}

\paragraph{} Les références se font sur des entiers uniquement. Elles sont implémentées en ajoutant un constructeur Ref au type Env.elt et sont stockées en tant que références de Caml dans l'environnement. Les trois opérations (déclaration, assignation et déréférencement) se font alors naturellement.

\paragraph{} Les tableaux ont été implémentés quasiment comme les références. Le type Env.elt a été enrichi d'un constructeur Array contenant un tableau. Les trois opérations (création de tableau, assignation à un indice, lecture d'une case) se font alors naturellement.




\section{La machine à pile SECD} % section à remplir par Léo

\subsection{Présentation}

La machine à pile est réalisée par la fonction step du fichier machine.ml . Cette fonction prend en argument une pile d'instruction machine (de type instruction list) et affiche un entier lorsqu'il a terminé. Chaque step lit et exécute l'instruction au sommet de la pile.

La compilation du code parsé est effectuée par la fonction build situé dans le même fichier. Cette fonction prend en argument un programme fouine parsé (de type programme) et renvoie une pile d'instruction machine (de type instruction list).

La machine comprend les indices de bruijn, calculée au moment de la compilation par la fonction \texttt{find\_bruijn}. Les ACCESS ne se font plus sur le nom de la variable mais sur la position dans l'environnement

\subsection{Implémentation}






\section{Interface et interprétation mixte}

\subsection{Pureté du code}

\subsection{Interprétation mixte}



\section*{Conclusion}

On constate que blibla.