\documentclass[xcolor=dvipsnames]{beamer}

\usepackage[utf8]{inputenc}
\usepackage{xcolor}

\usetheme{metropolis}

\title{Présentation finale projet 2 \\ ~ ~ Où l'on parle de rongeurs}
\author{Valque Léo - Guillaume Coiffier}
\institute{ENS de Lyon}
\date{2017}

\newcommand{\fouine}{\textsf{\textcolor{Orchid}{f}\textcolor{Cyan}{o}\textcolor{Orchid}{u}\textcolor{Cyan}{i}\textcolor{Orchid}{n}\textcolor{Cyan}{e} }}


\begin{document}

\maketitle

\begin{frame}
 \centering
 \Large un interpréteur \fouine $\quad \bigcup \quad $ une machine SECD
\end{frame}

\section{L'interpréteur fouine}

\begin{frame}
\frametitle{L'interpréteur fouine}
  Fonctionnalités de l'interpréteur
  \begin{itemize}
   \item \fouine pur \pause
   \item Exceptions \pause
   \item Références sur des entiers \pause
   \item Tableaux d'entiers \pause
  \end{itemize}
  Environnement utilisé : Table de hachage
\end{frame}

\begin{frame}
\frametitle{L'interpréteur fouine}
  Comment ça marche en pratique ? \\ ~ \\ \pause
  \begin{description}
   \item[Les exceptions :] On propage un booléen du \texttt{raise} au \texttt{try} juste au dessus \pause
   \item[Tableaux et références :] Stockés dans l'environnement
  \end{description}
\end{frame}

\begin{frame}
\frametitle{Interprétation mixte}
  Les fonctions et fonctions récursives sont pures. \pause
 \begin{itemize}
   \item Constructeur \texttt{Pure of programme} \pause
   \item Fonction \texttt{label\_pure\_code} qui insère les constructeurs purs \pause
   \item Interprétation quasiment normale \pause
   \item Recopie d'environnement de \fouine vers la machine
  \end{itemize}
\end{frame}

\section{La machine à pile SECD}

\begin{frame}
\frametitle{indices de bruijn}
  \begin{itemize}
    \item Dans la machine, on veut accèder aux élements par leurs position et non pas par leur nom \pause
    \item lorsqu'on compile le code, on stocke le nom des variables défini dans une pile \pause
    \item lorsqu'on compile l'access à un élement, on regarde quel est la position actuel de cette élement dans la pile. on remplace le nom par la valeur de cette position \pause
    \item let x = 2 in ( let y = 3 in x+y) + x devient \\
    let 2 in (let 3 in access(1)+access(0) ) + access(0)
\end{frame}

\begin{frame}
\frametitle{conversion d'environnement}
  \begin{itemize]
    \item Objectif : Traduire l'environnement de l'interpreteur vers l'environnement de la machine. \pause
    \item Problème : Elle sont très différentes. \pause
    \item dans l'interpreteur, c'est une table de hachage avec pour clés des strings. \pause
    \item dans la machine, c'est une simple liste avec pour "clé" des entiers (indices de bruijn). \pause 
\end{frame}

\begin{frame}
\frametitle{conversion d'environnement}
  \item On transmet pas directement l'environnement. \pause
  \item On transforme l'environnement de l'interpreteur en programme avec des let in qui met devant le code envoyer à la machine \pause
  \item La machine reçoit donc de l'environnement par ce biais \pause
  \item Problème :  c'est lent, la traduction de l'environnement est en O(n) où n est la taille de l'environnement
\end{frame}

\section*{Conclusion}
\begin{frame}{Conclusion}
C'était bien.
\end{frame}

\begin{frame}{Conclusion}
 \centering
 \Large Merci pour votre attention ! \\ ~ \\
 \includegraphics[width = 0.8\textwidth]{fouine_hiver.jpg}
\end{frame}


\end{document}



